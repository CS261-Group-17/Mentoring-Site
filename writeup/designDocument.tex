\documentclass[10pt]{article}

\usepackage[utf8]{inputenc}
\usepackage{amssymb}
\usepackage{enumitem}
\usepackage{longtable}
\usepackage{nopageno}
\usepackage{float}
\usepackage{graphicx}
\graphicspath{{./images/}}

\usepackage{rotating}
\def\rot{\rotatebox}

\usepackage{url}
\usepackage{hyperref}
\hypersetup{colorlinks=true, linkcolor=black, filecolor=magenta, urlcolor=blue,
    citecolor=black}

\usepackage[normalem]{ulem}
\useunder{\uline}{\ul}{}
\usepackage{multirow}

\usepackage{geometry}
\geometry{
 a4paper,
 total={170mm,257mm},
 left=10mm,
 right=10mm,
 top=10mm,
 bottom=10mm,
}


\begin{document}


\begin{center}
    \Huge\textbf{CS261 Coursework Design Document}\\
    \vspace{2mm}
    \large{\textit{\textbf{Group 17:} Ben Lewis, Dan Risk, Edmund Goodman,
    Jay Re Ng, John-Loong Gao, Rahul Vanmali, Tomás Chapmann Fromm}}
\end{center}


\section{Introduction}
% \vspace{-6mm}\section{Introduction}\vspace{-2mm}
Deutsche Bank requires a system prototype to support the mentoring process for
employees. The principal purpose of this document is to broadly record the
process documentation and design choices for said system prototype. For process
documentation, this document will detail development methodologies in addition
to workflow and project organisation. Design choices include a prototype user
interface design and technical diagrams demonstrating user interactions between
multiple users and the system. Furthermore, our testing strategy and technical
descriptions such as choice of technologies used and system architecture will be
covered in depth to illustrate a fuller idea of the final product.

\section{Process documentation and planning}
\subsection{Development methodology}
We opted for an agile development methodology based on the Scrum model. The
development timeline will be broken down into a large initial planning phase and
a series of week-long sprints. We chose to elect our project manager as scrum
master, since this role is similar to that of the project manager in organising
the team and improving workflow. We will use Trello to create and manage a scrum
board, which will allow for easy management of the product backlog. At the start
of each sprint, we will meet to discuss the items from the product backlog to be
completed in that sprint - this will usually be decided according to the project
timeline below, however some small adaptations may need to be made. Each day
within the sprint, the team will discuss progress and bring to attention any
challenges they are facing. At the end of the sprint, we will carry out a sprint
review, and each developer can showcase parts of the system which have been
completed. The sprint cycle is completed with a sprint retrospective, where
improvements can be made to the development process before the next sprint.

An agile methodology is optimal for a short timeframe as it allows rapid product
development with concurrent implementation and testing so we can make effective
use of the time available. It also allows for more flexibility if we face
unexpected delays during development, which is likely with an inexperienced
team. Producing a thorough plan before beginning the agile development process
is suitable since deliverable deadlines are fixed, so ensuring work is completed
on time is critical. Additionally, contact with the client is limited so
requirements are unlikely to change and the project will not deviate much from
the initial plan.


\subsection{Project organisation and communication}
We decided on a flat team hierarchy since we have similar levels of software
development experience, self-managed but with well defined responsibilities. To
communicate between meetings, we decided to use Discord, as it is already a
popular choice within the team and is easy to learn for new users. It offers
both text and voice chat as well as a number of features such as GitHub
integration, so we are notified via Discord when pull requests are made, for
example.

Discord can be used for online meetings however we aim to convene in-person at
least twice a week, as this enables us to share ideas much more easily and
maintain focus during meetings. At these meetings we can compare progress on the
system with our plan, and assign tasks based on documentation or what tasks need
doing in order to stay on schedule. We will use Trello to manage the product
backlog, and see which tasks are currently in progress, to ensure the project
will be completed by the deadline.


\subsection{Project schedule}
\begin{figure}[H]
    \centering
    \includegraphics[width=0.75\textwidth]{Timetable}
    \caption{Gantt chart of the project schedule.}
    \label{fig:project_schedule_gantt}
\end{figure}

The timeline outlines which features of the system will be implemented in each
weekly sprint, divided according to the team which is working on them, with
final product documentation separated as this is a collaborative effort by the
whole team. As we are following a Scrum methodology, task order may be adjusted
weekly, ie. in each sprint, so that the project remains on track, so which
features are completed in each week will most likely change. It also acts as a
reminder that the final report should be completed, when time is available,
concurrently with the project itself, so it doesn't need to be written in its
entirety in the few days before submission.

\subsection{Risk management}
% \begin{table}[]
\begin{longtable}{|p{0.15\linewidth}|p{0.15\linewidth}|p{0.1\linewidth}|p{0.08\linewidth}|p{0.15\linewidth}|p{0.15\linewidth}|p{0.08\linewidth}|}
\hline \rot{45}{\textbf{Risk}} & \rot{45}{\textbf{Impact}} & \rot{45}{\textbf{Likelihood}} & \rot{45}{\textbf{Severity}} & \rot{45}{\textbf{Mitigation}} & \rot{45}{\textbf{Contingency}} & \rot{45}{\textbf{Residual}} \\ \hline\hline
    Team members unable to work & Smaller team until the member can resume work                     & 2 & 7 & Team members take precautions                                 & Reduce scope and reallocate work across team  & 2 \\ \hline
    Task longer than expected   & Project delayed until task completed                              & 5 & 4 & Allocate buffer time to allow overrunning tasks               & Re-assign team members to load balance        & 4 \\ \hline
    Poor code quality           & System fails testing or linting procedures                        & 5 & 4 & Code review and automated testing through CI/CD               & Re-assign team memebrs to fix pair program    & 3 \\ \hline
    Requirements change         & Design and existing code must be changed                          & 3 & 8 & Use an agile methodology to facilitate requirements change    & Update project plan and proceed with new one  & 3 \\ \hline
    Code lost                   & Code must be re-written                                           & 1 & 7 & Use `git` as version control, and remote backup to GitHub     & Restore code from local or remote backups     & 1 \\ \hline
    Problems with dependencies  & Component of project fails external library or technology         & 2 & 4 & Choose technologies and dependencies carefully                & Find replacement library or technology        & 2 \\ \hline
    Scope creep                 & Addition of unnecessary features causing growth of project scale  & 7 & 4 & Include extra features in project timeline, and stick to it   & Drop lowest priority features                 & 5 \\ \hline
\end{longtable}
% \caption{}
% \label{tab:my-table}
% \end{table}

\section{Technical description}

\subsection{Scope}
Our system must satisfy a complex specification, so its development must be
carefully planned and executed. However, the goal is to create a prototype, not
a finished product. This means that some features that would be required in
production, but are either do not fit in the timescale for the prototype, e.g. ,
or would require integration from external systems, e.g. company calendar
applications, should not be implemented. Our requirements analysis fully defines
the scope of what we will implement.

\subsection{System attributes}

% TODO: THIS NEEDS TO BE UPDATED
This section will communicate the different qualities needed by software, what
is within our scope, how we have addressed them, and what we are not doing for
this prototype.\\ The end goal of the software engineering lifecycle is to end
with satisfied customers. As such, it is crucial that usability of the software
is given important consideration. Our goal is to ensure that the website is
intuitive to use, has little ambiguity and can be learnt ideally without
assistance even for non-technical users. There will be a wide coverage of
acceptance testing carried out to ensure this is achieved. Along with this, our
prototype aims, if possible, to make our system accessible to people with
disabilities with features such as enlarging text and higher contrast colours.
\\Compatibility and portability are important points to consider since we want to
make our system as portable as it needs to be, and compatible with systems used
by most of our users. Being a web app, makes it portable since every user will
have some access to a device able to run a browser. Extreme cases such as legacy
browsers will not be prioritised since it is unlikely to be used. The prototype
will not include desktop and mobile applications since it is out of the scope
and not needed with a web app, this early in the mentoring system. \\Security is a
relevant quality to consider, especially since we do not want to create issues
for any relevant parties involved. Complete and effective considerations to
security for most areas of vulnerability is out of scope for a prototype
implementing its core functionalities. Important security details such as
account access protection through salted/hashed stored passwords are to be
implemented. Data will only be accessed through the REST API, allowing us to
ensure only required data is sent, and adds a layer of security since our
database is not directly interacted with. Allows us to also validate any user
information before storing it in the database.

\subsection{Technologies used}

% NEED TO UPDATE THIS TO USE TABLE
A fundamental decision in the project is deciding which technologies should be
used to compose the so-called “tech stack”. Since this is such an important set
of decisions, we carefully considered a number of options with respect to a set
of criteria which assess the quality of a component technology choice. These
criteria include:
\begin{itemize}
    \item
        Suitability, how well does the technology match the problem
        our system solves?
    \item
        Documentation and popularity, does the choice have strong
        documentation and an active community likely to have already answered problems
        we may encounter?
    \item
        Consistency, how well the choice meshes with the other
        component technologies?
    \item
        Performance, will the technology run sufficiently fast,
        and can it be scaled later needing to switch to something else more performant?
    \item
        Experience, how much prior knowledge does our team have of the technology?
\end{itemize}

We decided to implement the project as a web application, as opposed to a mobile or
desktop only one. This is because it is generic and can be accessed by almost
all users, as essentially every modern device has a web browser. However, this
design choice must later be taken into consideration when ensuring the UI is
suitable for mobile. This choice was made as the system must be widely
accessible by users with many different types of devices, and a web application
was the only suitable choice for this.

We chose Django for our backend framework, which runs on top of Python. This is
suitable for our system as it facilitates fast development, which is required to
finish our prototype in the very short turnaround of around five weeks, with
very short code sprints, and it inherently supports our non-functional
requirements of security and scalability. Furthermore, it has robust
documentation and is widely used with an active community. Additionally, Python
is so ubiquitous that it can interface consistently with almost any other
technology choice, for example persistent data storage and sentiment analysis.
Finally, our team has a lot of experience with Python development, and
additionally testing techniques for Python, mitigating learning curves and
ensuring bug-free code.

Since we are making a web application, we will use HTML/CSS/JS. We decided to
use the Vue.JS framework for the frontend JS, as it facilitates writing
responsive websites, but has a much less steep learning curve than other
frameworks such as React, and is known to integrate well with Django, the
backend framework we propose using. Furthermore, we are going to use bootstrap
as the CSS framework as it allows us to quickly develop a professional looking
website with little hassle by providing building block CSS classes.

We chose PostgreSQL to store our persistent data in the form of a relational
database. This is because the type of data recorded by the system is well suited
to storage in a relational schema. For example, storing a set of users, each of
whom have various properties and relationships with other users is a very good
match for this model. Furthermore, postgresql has very good documentation in
comparison to other relational databases, and it meshes well into our tech stack
as django and postgresql are commonly used in tandem. Finally, all of our team
has experience with it due to the CS258 databases module.

We chose the python library NLTK (natural language toolkit) to implement our
sentiment analysis. This is because it is compatible with the rest of the
backend framework, which is already written in python, and it is the most
popular and widely used library within its field.

We will use JSON for the REST API, as it is incredibly ubiquitous, and links
easily into most web frameworks, as it is a subset of javascript. Additionally,
both our backend and frontend team members have experience with it, which is
crucial as it will form the link between the two sides of the technology stack,

We chose docker for our containerisation. This supports all of the rest of the
tech stack, and allows “lift-and-shift”, which is a very useful property for
prototype systems. Additionally, it is by far the most widely used choice for
containerisation, and has good documentation. Finally, none of our team has
experience with any form of containerisation, so choices cannot be compared by
this metric.

We chose git with GitHub for our version control. This is because it is the
industry standard for version control, every member of our team had at least
rudimentary experience with it from CS133, and some members of our team had
experience with advanced features for CI/CD such as GitHub Actions.

\subsection{System architecture}
Our system architecture follows the MVC (Model View Controller) architecture
which is a well-known model to be successful in creating a system that presents
data obtained through a controller that interacts with the database. This is
reflected in the docker containerisation figure with our website being the view,
the Django system being the controller, and Postgresql as the model. Being
modularised to 3 separate components, allows easy distribution of development,
with each focusing on a subset of skill sets. One downfall of this architecture
is that the system is tightly connected in that if one system fails, the rest
falls - however we believe that this is outweighed by the benefits of this
architecture. Furthermore, the fact that it is a trusted design pattern suggests
it is not an inherently problematic approach. A REST API will be used to serve
data to the user, allowing both static site pages and dynamic JSON data to be
served from the Django backend.

We decided to containerise our model, as the specification stated that Deutsche
Bank already uses containers for this type of application, so it will provide
modularity within their existing systems. Furthermore, it gives the property of
easy “lift-and-shift”, meaning that all the dependencies can be handled within
the containers, which themselves are portable and compatible across machines.

We decided against using microservices, instead opting for a monolithic
architecture. This was predominantly for two reasons: Django has a higher
overhead than other backend frameworks like Flask, so making multiple
microservices can reduce performance; and there is not a clean way nor a good
reason to split the internal logical model into separate microservices.

\begin{figure}[H]
    \centering
    \includegraphics[width=0.75\textwidth]{architecture}
    \caption{A system diagram of our system architecture.}
    \label{fig:system_architecture_diagram}
\end{figure}

The containerisation has two distinct parts. The first is the three containers
dynamically interact with each other based on user requests, with the NginX
container serving requests into the Django backend container, which runs the
internal logic, and can look up data from the postgresql database container. The
second is the final container which generates the static site content from
Vue.js to pure HTML, CSS, and Javascript. This is then copied into the Django
container so it only needs to be generated once, not on a request-by-request
basis.

\subsubsection{Database Design}

The User table stores relevant user information required by the overall system.
BusinessArea and StrengthWeakness tables lists all possible options the user may
choose as it is not free text. PasswordReset should only have unique userIDs to
ensure there is only one reset code, and should be removed if it has expired.
Notifications and Milestones will be searched by userID to obtain a user’s
relevant data entries. Authentication stores a token to be matched with what is
stored in the user's cookies to reduce the number of logins as it can be
annoying for users.
\begin{figure}[H]
    \centering
    \includegraphics[width=0.75\textwidth]{UserDB}
    \caption{User Schema.}
    \label{fig:User Schema}
\end{figure}

Pairing table reflects the current mentor-mentee pairings between users.
Important constraint is that a userID must be unique in the menteeID column
since mentee’s can only have one mentor. Event table is different from a meeting
in that it has multiple mentees, a capacity of number of people that can join
and a type – GroupSession or Workshop. ProposeMeeting can only store one
proposition for a meeting at a time relative to a pairing. If the mentor
rejects, it is reflected by a boolean and a note for the mentee.
RejectMentoringOffer is expected to be used in suggestion lists, to temporarily
hide a mentee in the case that mentee rejects a mentoring offer (F08).

\begin{figure}[H]
    \centering
    \includegraphics[width=0.75\textwidth]{MentoringDB}
    \caption{Mentoring Schema.}
    \label{fig:Mentoring Schema}
\end{figure}

Feedback has been decomposed to 4 types. To ensure efficiency and scalability,
each type of feedback has its own table rather than trying to abstract
commonalities, creating dependencies that could be problematic in the future.
The difference between MeetingFeedback and EventFeedback is if it’s for a
meeting or event. General Feedback stores overall feedback to a mentor by
mentee. These 3 will store an analysis of its sentiment, that will be used in
the suggestion process. SystemFeedback has two types, error or feature that is
used by developers to improve their system.

\begin{figure}[H]
    \centering
    \includegraphics[width=0.75\textwidth]{FeedbackDB}
    \caption{Feedback Schema.}
    \label{fig:Feedback Schema}
\end{figure}



\begin{figure}[H]
    \centering
    \includegraphics[width=0.75\textwidth]{ER}
    \caption{Entity Relationship Diagram.}
    \label{fig:Entity Diagram}
\end{figure}


\subsubsection{Object Structure}

% Believe I have updated this to the most recent one
\begin{figure}[H]
    \centering
    \includegraphics[width=0.75\textwidth]{Objects}
    \caption{A UML class diagram of our object structure.}
    \label{fig:uml_class_diagram}
\end{figure}


\subsubsection{API design}
We have planned to design an API which will be used by the front-end to
interface with the core back-end components of the system. We've decided to
settle on a REST API implementation, as it emphasises client-server decoupling,
which will make it easier to separate front-end and back-end tasks between our
developers—something which is of the utmost importance when working in a
smaller, less experienced team. The REST API system will give the front-end
developers access to fixed API endpoints, which will act as a bridge between the
front-end and back-end of the application. The front-end developers won't need
to know how these endpoints are implemented, they only need to know what
functionality they provide. This will save development time as developers can
now focus on their specialties. This modular way of designing the API allows for
back-end code to be modified without having to consult the front-end developers
about potential consequences. As long as the specification is followed, a
back-end modification should have no effect on how the front-end developers
interact with the system. Overall, this will save time as unnecessary
communication between developers will be eliminated.
% Edmund do it, long table

\section{UI/UX design}

\subsection{Page Hierarchy}
The page hierarchy below displays the sites that our webpage comprises of. Here
is a brief outline of what each page will consist of:
\begin{itemize}
    \item Login Screen: The landing page of the website that allows the user to
    login to go to the dashboard or the sign up page to create an account.
    \item Sign Up: Allows the user to create an account.
    \item Dashboard: The central hub for the user where all pages can be
    accessed from and displays useful information that may be useful (look at
    page designs for more detail).
    \item Profile: This page has 3 purposes: resetting the users password,
    seeing the users data and modifying the users data.
    \item Plans of Action: Displays the users plan of action and their mentees
    plans of actions too if they have any.
    \item Individual POA: If the user clicks on a POA, then they come to this
    page where the POA can be viewed and modified.
    \item Schedule: Shows the users schedule, allows them to request meetings
    and has a list of feedback and meeting requests that the user should respond
    to.
    \item Group Events: Allows the user to search for workshops and group
    sessions within the system that they can attend.
    \item Individual Event: A page that details an event (meeting or group
    event) and allows the owner of an event to modify or delete the event.
\end{itemize}

\begin{figure}[H]
    \centering
    \includegraphics[width=0.75\textwidth]{Hierarchy}
    \caption{A diagram of website page hierarchy.}
    \label{fig:website_page_hierarchy}
\end{figure}


\subsection{User-system interaction}
The specification was also very vague about how to match up mentors and mentees,
therefore this diagram clarifies all the ambiguities around this.
\begin{itemize}
    \item The user can only have 1 mentor at a time
    \item The system can give the user a list of potential other users who are currently looking for mentees
    \item The user then picks one of these users and a mentor mentee request is sent to the potential mentor
    \item If the potential mentor accepts, then a relationship is formed
    \item Otherwise, the mentee needs to choose another mentor from the list
\end{itemize}

\begin{figure}[H]
    \centering
    \includegraphics[width=0.75\textwidth]{MentorMentee}
    \caption{A diagram of the states required to match a mentor and a mentee.}
    \label{fig:state_mentor_mentee_diagram}
\end{figure}

The diagram below displays what functions the Schedule page has. This handles a
lot of ambiguities around how the meeting and feedback system will work within
our system. The important points from this diagram is that:
\begin{itemize}
    \item Mentee's can cancel meetings, but mentors can only reschedule
    \item Mentee's can request a meeting, but mentors cannot not
    \item For a meeting to be officially approved, both parties need to agree on the time and date otherwise they go into this loop of suggesting times to each other
    \item The user will be prompted on the schedule page to give feedback to recent events they have gone to
\end{itemize}

\begin{figure}[H]
    \centering
    \includegraphics[width=0.75\textwidth]{Meeting}
    \caption{A diagram of the states required to schedule a meeting.}
    \label{fig:state_meeting_diagram}
\end{figure}

\subsection{Page Design}
The dashboard is the central page for users that displays all the useful
information for the user, such as their upcoming meetings, their current plan of
action and allows them to get a new mentee or mentor. The navigation bar at the
top of the page will be on most of the pages and will allow the user to access
all the other pages of the website as well as their notifications. The
notifications inform the user about any recent events that they may be
interested in such as giving feedback about a recent event or a mentee/mentor
meeting coming up. The drop down button next to the notification bell will link
to the user's profile page, the website feedback form and a sign out button.

\begin{figure}[H]
    \centering
    \includegraphics[width=0.75\textwidth]{Dashboard}
    \caption{The initial design of our system dashboard.}
    \label{fig:dashboard_design}
\end{figure}

The schedule page will have the users calender and all the requests the user has
(meeting and feedback). This user is not a mentee so the request meeting button
at the bottom is currently missing, but if the user was a mentee then there
would be a request meeting button at the bottom. Additionally, clicking on a
meeting will take the user to the individual event page for the meeting which
will have the feedback log for the meeting.

\begin{figure}[H]
    \centering
    \includegraphics[width=0.75\textwidth]{Schedule}
    \caption{The initial design of our system schedule page.}
    \label{fig:dashboard_design}
\end{figure}

\section{Testing}
\subsection{Test cases}
% Need to use a long table, Edmund
\subsection{Unit and integration testing}
Tests will form an integral part of our development cycle, as we plan to follow
the agile test driven development ideologies. We will implement this by creating
unit tests for all of the above test cases, allowing us to ensure all the
required properties of the system are met. These can then be run throughout the
development process. All of the tests will be automatically run on a pull
request to the GitHub repository, and must pass in order for the code to be
merged into the main branch, to ensure its validity. This is an implementation
of “continuous integration” from CI/CD, and will be implemented using GitHub
Actions.

There are many tools for unit testing which can be used. Since we are using a
python backend framework, we will use `PyTest' for unit tests on the backend
code, along with `codecov' to assess the coverage of the tests, and `pylint' to
ensure that stylistic code is being written. Since we are using a javascript
frontend framework, we will use `Jest' for unit tests on the frontend code and
UI, along with `ESLint' to check that the Javascript is correct. Finally, we
will use Postman to test that the API serves the correct data. There are
relatively few tools for integration testing, and it will likely mostly have to
be done by hand. We will use the `Puppeteer' tool to automate this task as far
as possible. The combination of all of these tools allows us to create a testing
suite which will fully cover our system.


\bibliographystyle{ieeetr}
\bibliography{designDocument.bib}


\end{document}
