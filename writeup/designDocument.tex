\documentclass[10pt]{article}

\usepackage[utf8]{inputenc}
\usepackage{amssymb}
\usepackage{enumitem}
\usepackage{longtable}
\usepackage{nopageno}
\usepackage{float}

\usepackage{rotating}
\def\rot{\rotatebox}

\usepackage{url}
\usepackage{hyperref}
\hypersetup{colorlinks=true, linkcolor=black, filecolor=magenta, urlcolor=blue,
    citecolor=black}

\usepackage[normalem]{ulem}
\useunder{\uline}{\ul}{}
\usepackage{multirow}

\usepackage{geometry}
\geometry{
 a4paper,
 total={170mm,257mm},
 left=10mm,
 right=10mm,
 top=10mm,
 bottom=10mm,
}


\begin{document}


\begin{center}
    \Huge\textbf{CS261 Coursework Design Document}\\
    \vspace{2mm}
    \large{\textit{\textbf{Group 17:} Ben Lewis, Dan Risk, Edmund Goodman,
    Jay Re Ng, John-Loong Gao, Rahul Vanmali, Tomás Chapmann Fromm}}
\end{center}


\section{Introduction}
% \vspace{-6mm}\section{Introduction}\vspace{-2mm}
Deutsche Bank requires a system prototype to support the mentoring process for
employees. The principal purpose of this document is to broadly record the
process documentation and design choices for said system prototype. For process
documentation, this document will detail development methodologies in addition
to workflow and project organisation. Design choices include a prototype user
interface design and technical diagrams demonstrating user interactions between
multiple users and the system. Furthermore, our testing strategy and technical
descriptions such as choice of technologies used and system architecture will be
covered in depth to illustrate more full idea of the final product.

\section{Process documentation and planning}
\subsection{Development methodology}
We opted for an agile development methodology based on the Scrum model. The
development timeline will be broken down into a large initial planning phase and
a series of week-long sprints. We will have multiple meetings each sprint cycle
to assess progress (sprint review) and adapt the plan if required.

An agile methodology is optimal for a short timeframe as it allows rapid product
development with concurrent implementation and testing so we can make effective
use of the time available. It also allows for more flexibility if we face
unexpected delays during development, which is likely with an inexperienced
team. Producing a thorough plan before beginning the agile development process
is suitable since deliverable deadlines are fixed, so ensuring work is completed
on time is critical. Additionally, contact with the client is limited so
requirements are unlikely to change and the project will not deviate much from
the initial plan.

\subsection{Project organisation and communication}
We decided on a flat team hierarchy since we have similar levels of software
development experience, self-managed but with well defined responsibilities. To
communicate between meetings, we decided to use Discord, as it is already a
popular choice within the team and is easy to learn for new users. It offers
both text and voice chat as well as a number of features such as GitHub
integration, so we are notified via Discord when on changes, for example when
pull requests are made.

Discord can be used for online meetings however we aim to convene in-person at
least twice a week, as this enables us to share ideas much more easily and
maintain focus during meetings. At these meetings we can compare progress on the
system with our plan, and assign tasks based on documentation or what tasks need
doing in order to stay on schedule. We will use Trello to assist in the task
management process, as it provides a visual representation of current project
progress.

\subsection{Project schedule}

\subsection{Risk management}

% \begin{table}[]
\begin{longtable}{|p{0.15\linewidth}|p{0.15\linewidth}|p{0.1\linewidth}|p{0.08\linewidth}|p{0.15\linewidth}|p{0.15\linewidth}|p{0.08\linewidth}|}
\hline \rot{45}{\textbf{Risk}} & \rot{45}{\textbf{Impact}} & \rot{45}{\textbf{Likelihood}} & \rot{45}{\textbf{Severity}} & \rot{45}{\textbf{Mitigation}} & \rot{45}{\textbf{Contingency}} & \rot{45}{\textbf{Residual}} \\ \hline\hline
    Team members unable to work & Smaller team until the member can resume work                     & 2 & 7 & Team members take precautions                                 & Reduce scope and reallocate work across team  & 2 \\ \hline
    Task longer than expected   & Project delayed until task completed                              & 5 & 4 & Allocate buffer time to allow overrunning tasks               & Re-assign team members to load balance        & 4 \\ \hline
    Poor code quality           & System fails testing or linting procedures                        & 5 & 4 & Code review and automated testing through CI/CD               & Re-assign team memebrs to fix pair program    & 3 \\ \hline
    Requirements change         & Design and existing code must be changed                          & 3 & 8 & Use an agile methodology to facilitate requirements change    & Update project plan and proceed with new one  & 3 \\ \hline
    Code lost                   & Code must be re-written                                           & 1 & 7 & Use `git` as version control, and remote backup to GitHub     & Restore code from local or remote backups     & 1 \\ \hline
    Problems with dependencies  & Component of project fails external library or technology         & 2 & 4 & Choose technologies and dependencies carefully                & Find replacement library or technology        & 2 \\ \hline
    Scope creep                 & Addition of unnecessary features causing growth of project scale  & 7 & 4 & Include extra features in project timeline, and stick to it   & Drop lowest priority features                 & 5 \\ \hline
\end{longtable}
% \caption{}
% \label{tab:my-table}
% \end{table}

\subsection{System attributes}

\section{Technical description}

\subsection{Scope}
Our system must satisfy a complex specification, so its development must be
carefully planned and executed. However, the goal is to create a prototype, not
a finished product. As such, various aspects which might be required for a
system going into production are not needed within a prototype.

One example of this is limitations on the scalability, for example the number of
concurrent users. It is much more important that key features are implemented
than being able to support dynamically changing high numbers of concurrent
users, as the prototype will likely only be tested by a few users at any time.
However, this does not mean that the prototype should not be designed without a
good basis for facilitating scalability later as the system moves through its
lifecycle. Another example of this is the use of Deutsche Bank branding, which
is explicitly not to be used, although it would be incorporated in the
production system. Instead, a generic and simple theme should be used in the
prototype, which could then be later replaced with the correct branding.

\subsection{Technologies used}

\subsection{System architecture}

\subsubsection{Data model}
\subsubsection{Database design}
\subsubsection{API design}

\section{UI/UX design}

\subsection{Site design}
\subsubsection{Page design}
\subsubsection{Page heirarchy}
\subsubsection{UI design principles and accessibility}

\subsection{User-system interaction}
\subsubsection{Use case diagram}
\subsubsection{Sequence diagram}

\section{Testing}
\subsection{Test cases}
\subsection{Unit and integration testing}
\subsection{Acceptance testing}

\bibliographystyle{ieeetr}
\bibliography{designDocument.bib}


\end{document}
