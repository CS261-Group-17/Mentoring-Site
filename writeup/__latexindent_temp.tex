\documentclass[10pt]{article}

\usepackage[utf8]{inputenc}
\usepackage{amssymb}
\usepackage{enumitem}
\usepackage{nopageno}
\usepackage{geometry}
\geometry{
 a4paper,
 total={170mm,257mm},
 left=10mm,
 right=10mm,
 top=10mm,
 bottom=10mm,
}

\begin{document}


\begin{center}
    \Huge\textbf{CS260 Coursework 2}\\
\end{center}


\vspace{-6mm}\section{Introduction}\vspace{-2mm}
This document indicates our current understanding of the proposed project, and
will be used alongside Deutsche Bank to clarify our position. We detail and
analyse the requirements for a proposed mentoring software platform which should
function internally at Deutsche Bank. Many employees want a broader
understanding of how different departments within the company work in unison to
successfully complete business projects. Alongside this, employees may also have
weaknesses in certain business areas, which could be improved by working with
more experienced people in different departments. A robust mentoring software
platform would solve these issues, by appropriately assigning users
knowledgeable mentors who could help them at a personal and professional level.

\vspace{-4mm}\section{Glossary}\vspace{-2mm}
\begin{itemize}[leftmargin=1.2cm,noitemsep,align=left]
    \item
        \textbf{Mentor:} The users who can approve a requested meeting and
        select confirm a date.
    \item
        \textbf{Mentee:} The users who can request a meeting in a series of
        targeted dates.
    \item
        \textbf{Plan of action:} The future task and achievement milestones that
        can be created by mentee and mentor, and mentee can mark it as complete
        or not.
    \item
        \textbf{Meeting:} This is the event that is held by a mentor and a mentee
        in a specific time duration. Mentors teach business knowledge to
        mentees. After the meeting, all users are available to provide feedback.
    \item
        \textbf{Workshop:} This is an event that is held by multiple people. The
        main purpose is to major in mentees' weaknesses.
    \item
        \textbf{Feedback:} Mentor and mentee can summarise a review with each
        other, based on their behaviours and sentiment during the whole training
        period.
    \item
        \textbf{Group session:} This function is nearly the same as a workshop,
        except there is no set topic.
    \item
        \textbf{Event:} A broad term that means meeting, workshop or group session.
\end{itemize}

\vspace{-8mm}\section{User stories}\vspace{-2mm}
User A has recently graduated from university, and has joined Deutsche Bank in
the Human Resources department. Coming from a non-technical background and being
thrown into a very tech focused environment has led the user to believe it would
be beneficial for them to have a general idea of the technologies they were
using worked under the hood. They would therefore like to be paired with an
experienced mentor who knows the inner workings of Deutsche Bank's technology
system.

User B has been working as a manager at Deutsche Bank for 7 years, and has
mentored many employees throughout the years. The user notes that many of the
mentees have a keen interest in how conflict in the workplace is handled from a
managerial position. Rather than having to teach the same principles repeatedly
for each mentee, the user would like a system that would allow them to create
group sessions where these management principles could be taught to multiple
mentees at once.

User C has recently started mentoring employees at Deutsche Bank, and would like
to do as much as possible to ensure their mentoring actually provides value to
the mentees. Whilst the user believes the sessions are productive, there's no
formal way for them to attain mentee feedback. The user would therefore like a
system where mentees could provide feedback after each session. This would then
allow the user to make changes about the way they mentor, to maximise their
value to each employee.

\vspace{-4mm}\section{Functional requirements}\vspace{-2mm}
\begin{table}[]
\begin{tabular}{|l|l|}
\hline
Functional requirements & Justification \\ \hline
                        &               \\ \hline
                        &               \\ \hline
                        &               \\ \hline
                        &               \\ \hline
                        &               \\ \hline
                        &               \\ \hline
                        &               \\ \hline
                        &               \\ \hline
                        &               \\ \hline
\end{tabular}
\caption{}
\label{tab:my-table}
\end{table}

\vspace{-4mm}\section{Non-functional requirements}\vspace{-2mm}
Table

\vspace{-4mm}\section{Team organisation}\vspace{-2mm}
\subsection{Team roles}\vspace{-2mm}
Table

\vspace{-4mm}\subsection{Scheduling}\vspace{-2mm}
In-person group meetings Monday and Friday, as well as additional meetings or
collaborative work time to help meet deadlines. Monday meetings to see progress
on weekend tasks and set tasks for the week, Friday meetings to see progress on
week and set tasks for the weekend.

\vspace{-4mm}\subsection{References}\vspace{-2mm}
References

\end{document}
