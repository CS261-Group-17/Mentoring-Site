\documentclass[10pt]{article}

\usepackage[utf8]{inputenc}
\usepackage{amssymb}
\usepackage{enumitem}
\usepackage{longtable}
\usepackage{nopageno}
\usepackage{float}

\usepackage{url}
\usepackage{hyperref}
\hypersetup{colorlinks=true, linkcolor=black, filecolor=magenta, urlcolor=blue,
    citecolor=black}

\usepackage[normalem]{ulem}
\useunder{\uline}{\ul}{}
\usepackage{multirow}

\usepackage{geometry}
\geometry{
 a4paper,
 total={170mm,257mm},
 left=10mm,
 right=10mm,
 top=10mm,
 bottom=10mm,
}

\begin{document}


\begin{center}
    \Huge\textbf{CS261 Coursework Design Document}\\
    \vspace{2mm}
    \large{\textit{\textbf{Group 17:} Ben Lewis, Dan Risk, Edmund Goodman,
    Jay Re Ng, John-Loong Gao, Rahul Vanmali, Tomás Chapmann Fromm}}
\end{center}


\vspace{-6mm}\section{Introduction}\vspace{-2mm}
This document indicates our current understanding of the proposed project, and
will be used alongside Deutsche Bank to clarify our position. We detail and
analyse the requirements for a proposed mentoring software platform which should
function internally at Deutsche Bank. Many employees want a broader
understanding of how different departments within the company work in unison to
successfully complete business projects. Alongside this, employees may also have
weaknesses in certain business areas, which could be improved by working with
more experienced people in different departments. A robust mentoring software
platform would solve these issues, by appropriately assigning users
knowledgeable mentors who could help them at a personal and professional level.



\bibliographystyle{ieeetr}
\bibliography{designDocument.bib}


\end{document}
