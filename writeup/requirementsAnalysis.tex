\documentclass[10pt]{article}

\usepackage[utf8]{inputenc}
\usepackage{amssymb}
\usepackage{enumitem}
\usepackage{longtable}
\usepackage{nopageno}
\usepackage{float}

\usepackage[normalem]{ulem}
\useunder{\uline}{\ul}{}
\usepackage{multirow}

\usepackage{geometry}
\geometry{
 a4paper,
 total={170mm,257mm},
 left=10mm,
 right=10mm,
 top=10mm,
 bottom=10mm,
}

\begin{document}


\begin{center}
    \Huge\textbf{CS260 Coursework 2}\\
\end{center}


\vspace{-6mm}\section{Introduction}\vspace{-2mm}
This document indicates our current understanding of the proposed project, and
will be used alongside Deutsche Bank to clarify our position. We detail and
analyse the requirements for a proposed mentoring software platform which should
function internally at Deutsche Bank. Many employees want a broader
understanding of how different departments within the company work in unison to
successfully complete business projects. Alongside this, employees may also have
weaknesses in certain business areas, which could be improved by working with
more experienced people in different departments. A robust mentoring software
platform would solve these issues, by appropriately assigning users
knowledgeable mentors who could help them at a personal and professional level.

\vspace{-4mm}\section{Glossary}\vspace{-2mm}
\begin{itemize}[leftmargin=1.2cm,noitemsep,align=left]
    \item
        \textbf{Mentor:} The users who can approve a requested meeting and
        select confirm a date.
    \item
        \textbf{Mentee:} The users who can request a meeting in a series of
        targeted dates.
    \item
        \textbf{Plan of action:} The future task and achievement milestones that
        can be created by mentee and mentor, and mentee can mark it as complete
        or not.
    \item
        \textbf{Meeting:} This is the event that is held by a mentor and a mentee
        in a specific time duration. Mentors teach business knowledge to
        mentees. After the meeting, all users are available to provide feedback.
    \item
        \textbf{Workshop:} This is an event that is held by multiple people. The
        main purpose is to major in mentees' weaknesses.
    \item
        \textbf{Feedback:} Mentor and mentee can summarise a review with each
        other, based on their behaviours and sentiment during the whole training
        period.
    \item
        \textbf{Group session:} This function is nearly the same as a workshop,
        except there is no set topic.
    \item
        \textbf{Event:} A broad term that means meeting, workshop or group session.
\end{itemize}

\vspace{-8mm}\section{User stories}\vspace{-2mm}
User A has recently graduated from university, and has joined Deutsche Bank in
the Human Resources department. Coming from a non-technical background and being
thrown into a very tech focused environment has led the user to believe it would
be beneficial for them to have a general idea of the technologies they were
using worked under the hood. They would therefore like to be paired with an
experienced mentor who knows the inner workings of Deutsche Bank's technology
system.

User B has been working as a manager at Deutsche Bank for 7 years, and has
mentored many employees throughout the years. The user notes that many of the
mentees have a keen interest in how conflict in the workplace is handled from a
managerial position. Rather than having to teach the same principles repeatedly
for each mentee, the user would like a system that would allow them to create
group sessions where these management principles could be taught to multiple
mentees at once.

User C has recently started mentoring employees at Deutsche Bank, and would like
to do as much as possible to ensure their mentoring actually provides value to
the mentees. Whilst the user believes the sessions are productive, there's no
formal way for them to attain mentee feedback. The user would therefore like a
system where mentees could provide feedback after each session. This would then
allow the user to make changes about the way they mentor, to maximise their
value to each employee.


\vspace{-4mm}\section{Functional requirements}\vspace{-2mm}                         % TODO: Add some waffle about process and italicising developer requirements. Put non-functional first?
\begin{longtable}{|p{0.55\linewidth}|p{0.4\linewidth}|}
    \hline
    \textbf{Functional requirements}
        &
    \textbf{Justification}
    \\ \hline\hline

    \textbf{F00 (Must Have) }
    The user can register an account.
    \textit{The user will be able to create an account by providing an email and
    password, and add data about themselves.}
        &
    A user account system is necessary for users to identify themselves and to
    access the application's functionality.
    \\ \hline

    \textbf{F01 (Could Have) }
    The website will teach new users with tutorials.
    \textit{The first time after the account is created, a tutorial for the
    major features of the website will be provided.}
        &
    Users are provided enough information to fully understand the application's
    features and functionality.
    \\ \hline

    \textbf{F02 (Must Have) }
    The user will be able to log into their account using their email and
    password and reset their password via email if they need to.
    \textit{The system will check the database to see if a matching account
    exists, and send a password reset link if required.}
        &
    User accounts are required to match up individuals. Being able to reset the
    password by email is important as people often forget their passwords.
    \\ \hline

    \textbf{F03 (Must Have) }
    The user will have a profile page, allowing them to view and modify data.
    \textit{Data includes: their email address; their name; their job title;
    whether they want to be a mentor, mentee, or both; their business area; 1-5
    strengths; and 1-5 weaknesses. A “save” button will be used to update any
    modified data and ensure that no essential data is left empty.}
        &
    Users must be able to update their profile page to keep the system
    up-to-date, and the profiles must include their basic information along with
    weaknesses and strengths in order for the system to suggest possible mentor
    and mentee pairing.
    \\ \hline

    \textbf{F04 (Must Have) }
    Any changes to the profile which will break the rules of mentoring between
    current relationships will notify the user.
    \textit{The notice is a warning and still allows the user to break the
    relationship.}
        &
    This forms a good compromise between maintaining the rules of mentoring,
    and not accidentally breaking good relationships
    \\ \hline

    \textbf{F05 (Must Have) }
    Allow mentees to request a mentor, who will have strengths matching their
    weaknesses.
    \textit{The system will add the user to the list of people looking for
    mentors.}
        &
    In order for mentors to be paired with mentees, mentees must first seek a
    mentor by specifying an area of weakness.
    \\ \hline

    \textbf{F06 (Must Have) }
    When selecting a mentee, a mentor will get access to their profile to help
    the mentor make a decision.
    \textit{Data will include their name, strengths, weaknesses, job title and
    business area.}
        &
    This allows the mentor to gauge which mentee is most suitable to them,
    adding a human element to the matching process.
    \\ \hline

    \textbf{F07 (Must Have) }
    Mentors will be able to select mentees to mentor from a list of mentees who
    satisfy the rules of mentoring, ordered by the likelihood of them being a
    good match with the mentor.
    \textit{The list will be filtered to exclude invalid mentees, e.g. same
    business area, and the total order for the list of mentees to select from
    will be based on matching metrics using data stored by the system.}             % TODO: Matching metrics comment?
        &
    This complements the above requirement by supplementing the human element of
    matching with a computational one which filters invalid matches, and
    encourages ones which would be good based on feedback and aggregated
    metrics.
    \\ \hline

    \textbf{F08 (Must Have) }
    When a mentor selects a mentee, the mentee will be able to accept or reject
    the offer of mentoring.
    \textit{The mentor profile is shown, and, if rejected, the mentee will then
    not be shown to that mentor for a specified period of time.}
        &
    This allows mentees to view the profiles from their available mentors and
    choose who they see the most fit, and avoids mentors repeatedly requesting
    mentees in a short time.
    \\ \hline

    \textbf{F09 (Must Have) }
    Mentors will be prompted to select mentees if they are not mentoring anyone.
    Mentees will only be allowed to have one mentor at a time.
    \textit{Mentors can have several mentees if they want, however the system
    must follow the rules of mentoring.}
        &
    Mentors should always have someone to mentor if there is any mentee
    compatible with them, as otherwise there may not be enough mentors.
    \\ \hline

    \textbf{F10 (Must Have) }
    Both mentors and mentees will be able to terminate the relationship at any
    time, and feedback should be required for why it was terminated.
    \textit{This feedback includes a numerical rating and text to be shown to
    the other person.}
        &
    Termination of any mentor/mentee relationships can be due to reasons of poor
    fit. Feedback is also necessary for mentors to improve their mentorship.
    \\ \hline

    \textbf{F11 (Must Have) }
    Mentees will be able to propose a meeting, providing a brief description of
    the agenda of the meeting, and/or the category it relates to.
    \textit{The agenda will be free text whereas the category will be from the
    discrete list of strengths/weaknesses.}
        &
    This follows the first rule of mentoring where the mentee must drive the
    relationship whenever they need unblocking.
    \\ \hline

    \textbf{F12 (Should Have) }
    Mentors will be prompted to suggest three meeting start/end times that would
    work for them (and info about where it will be).
    \textit{Must ensure that the meetings are not at the same time as another
    event in the system.}
        &
    This is to allow for flexibility in meeting time proposals by the mentor,
    whilst still following the rule of mentors giving up time/
    \\ \hline

    \textbf{F13 (Must Have) }
    Mentees will be able to either accept one of the meeting start/end times, or
    send a request back to the mentor asking for more meeting start/end times.
    \textit{The mentor must give up time to the mentee so this cycle is required.}
        &
    For a meeting to go ahead, both parties must be free for it. This confirms
    this, given the above requirement of mentors suggesting times for meetings.
    \\ \hline

    \textbf{F14 (Must Have) }
    After the end time of the meeting has passed, both users will be prompted to
    provide feedback on the meeting.
    \textit{They will summarise the meeting with a free text field, and numeric
    metrics, and whether the meeting was attended and be able to see the other's
    feedback.}
        &
    This is to allow the recording of discussions and for the users to reflect
    on the meeting where necessary.
    \\ \hline

    \textbf{F15 (Must Have) }
    Plans of action for a mentor/mentee pair will be able to be created by both
    parties.
    \textit{The system only allows for plans of actions to be created between a
    pair of people.}
        &
    This is suggested by the mentor and followed through by the mentee when
    possible to resolve the weaknesses of the mentee.
    \\ \hline

    \textbf{F16 (Must Have) }
    New milestones will be able to be added and modified by both parties, being
    either personal or professional, having a description, a priority, and a
    target completion date.
    \textit{The description will be free text, priority and a number and
    completion date.}
        &
    Plans of action have milestones to keep track of the progress made by the
    mentee. It is important for the mentee to have set goals and priorities to
    work towards.
    \\ \hline

    \textbf{F17 (Must Have) }
    Milestones will be able to be marked as completed by mentees, and if a
    milestone has not been completed after its target completion date, mentees
    will be prompted to mark it as complete or update its target date.
    \textit{This prompt will be given the next time the mentee signs in.}
        &
    This ensures that the mentee is able to see their progress, and the mentor
    can provide further guidance to the mentee if they remain on a milestone for
    a prolonged period.
    \\ \hline

    \textbf{F18 (Should Have) }                                                     % TODO: This is fulfilling the spec, so really should be must have?
    General feedback will be able to be provided at any time.
    \textit{Including free text submissions and updateable numerical metrics,
    and feedback for specific topic areas will be provided in the same way to
    refine the suggestion process.}
        &
    This is so feedback can be provided outside of just meetings and termination
    of relationships.
    \\ \hline

    \textbf{F19 (Should Have) }
    If there are enough people with a specified weakness, people identified as
    experts in that area will be prompted to organise a workshop on it.
    \textit{Using prior feedback, the system can determine if a person is an
    expert.}
        &
    Experts are able to run workshops to cover a specific 'weakness' topic
    extensively and potentially through a group setting.
    \\ \hline

    \textbf{F20 (Must Have) }
    To create a workshop, the user will set a workshop start/end, and a brief
    description of what the workshop will cover within the specified weakness
    category.
    \textit{The workshop will then be registered as an event within the system
    that anyone can go to.}
        &
    Information on the workshops can be viewed on the dashboard on the site. The
    description should provide attendees an idea of what the workshop will focus
    on.
    \\ \hline

    \textbf{F21 (Must Have) }
    Group sessions will be created, run, and reviewed in the same way as
    workshops.
        &
    Group sessions are equivalent to workshops within the system, with the one
    difference outlined in the glossary not being in the system.
    \\ \hline

    \textbf{F22 (Must Have) }
    Signed in users will be able to submit feedback/bug reports to the system.
    \textit{There will be a button at the top of the dashboard that allows you
    to go to the feedback screen.}
        &
    Feedback/bug reports is crucial to the maintenance and further development
    of the application.
    \\ \hline

    \textbf{F23 (Should Have) }
    The user will be able to disclose security issues.
    \textit{There will be a security.txt file for responsible disclosure of
    security issues - not necessarily requiring the user to be signed in.}
        &
    System administrators must be made aware of any security issues, so they can
    resolve them promptly/
    \\ \hline

    \textbf{F24 (Must Have) }
    The system will have a dashboard that gives a personalised view of the users
    current meetings, plans of actions, workshops/group sessions and access to
    the profile page.
    \textit{This will be the main page the user gets placed on when they log in.}
        &
    Users must be able to navigate the site easily, so a dashboard where
    important notifications are shown will facilitate this.
    \\ \hline

    \textbf{F25 (Must Have) }
    Users will be able to see a list of current workshops.
    \textit{It will be an ordered list of workshops grouped into weeks and then
    ordered by rating, which can be filtered by the users weaknesses.}
    &
    This will allow users to easily select workshops most relevant to them.
    \\ \hline

\end{longtable}


\vspace{-4mm}\section{Non-functional requirements}\vspace{-2mm}
\begin{longtable}{|p{0.55\linewidth}|p{0.4\linewidth}|}
    \hline
    \textbf{Non-functional requirements}
        &
    \textbf{Justification}
    \\ \hline\hline

    \textbf{NF00 (Must Have) }
    The system will be intuitive to use and navigate.
    \textit{The system must have a UI that makes it inherently clear to the user
    what all parts of the interface do. Each part of the UI will have a clear,
    distinct and useful purpose that is unambiguous.}
        &
    The user is able to clearly identify the functionalities of each widget
    through concise and thorough UI design.
    \\ \hline

    \textbf{NF01 (Must Have) }
    The system will be simple to use even for users with less technical experience.
    \textit{The UI and the tutorial will make the system suitable for everyone.}
        &
    The user does not need any familiarity with the application before beginning
    to use it.
    \\ \hline

    \textbf{NF03 (Should Have) }
    The system will be quick to respond to the user.
    \textit{90th percentile of API request times are under 10 seconds.}
        &
    The user should not feel any notable delay in the responsiveness of the
    system.
    \\ \hline

    \textbf{NF04 (Won't Have) }
    The system will be able to handle large and varied numbers of users.
    \textit{This will not be explicitly supported, but design choices will be
    made to facilitate it in future development}
        &
    This is out of scope of the prototype, however, it is useful to design so
    it can be easily scaled by future maintainers
    \\ \hline

    \textbf{NF05 (Must Have) }
    The system will be kept up to date.
    \textit{The system will be easy to maintain and modularised so that each        % TODO: Not sure how relevant this is?
    section of the software can independently be worked upon and inserted back
    into the system when doing maintenance. All code should be appropriately
    commented.}
        &
    The system should be regularly updated to improve performance and
    responsiveness.
    \\ \hline

    \textbf{NF06 (Must Have) }
    The system will be easy to test and validate its properties.
    \textit{The system will be modularised so that it can be tested on a system
    scale and so that each independent unit can be unit tested. Acceptance
    testing for the usability of the UI should be possible. CI/CD will be used
    to facilitate this.}
        &
    The system must be tested and validated to ensure the functionalities and UI
    work correctly and are following its intended design.
    \\ \hline

    \textbf{NF07 (Must Have) }
    The system will follow all relevant laws.
    \textit{The system will comply with GDPR, cookie laws and any other relevant
    laws.}
        &
    It is paramount that the system is legal for it to be used in a corporate
    setting.
    \\ \hline

    \textbf{NF08 (Must Have) }
    The site will employ good security practises.
    \textit{Such as hashing/salting passwords and sanitising user input.}
        &
    Security is crucial to ensuring that any registered user's data is protected.
    \\ \hline

\end{longtable}

\vspace{-4mm}\section{Team organisation}\vspace{-2mm}
\subsection{Team roles}\vspace{-2mm}

\begin{table}[H]
\begin{tabular}{p{0.2\linewidth}|p{0.1\linewidth}|p{0.6\linewidth}}
    \hline
    \textbf{Team member} & \textbf{Role}                      & \textbf{Responsibilities} \\ \hline\hline
    Dan Risk             & Project manager                    &
        Schedule meetings, coordinate tasks for group members to balance team
        workload and ensure the project is on schedule to meet the deadline.
    \\ \hline
    Ben Lewis            & \multirow{3}{1\linewidth}{Website developer} & \multirow{3}{1\linewidth}{
        Design front-end website layout and functions, interpret information
        sent from back-end and display it to the user.
    } \\
    Jay Re Ng            & \\
    John-Loong Gao       & \\ \hline
    Edmund Goodman       & \multirow{3}{1\linewidth}{Back-end engineer} & \multirow{3}{1\linewidth}{
        Implement the system to interpret and process data received from the
        front-end, provide data to be displayed to the front-end, and manage
        effective data storage in the system database.
    } \\
    Tomas Chapman Fromm  & \\
    Rahul Vanmali        & \\
    \hline
\end{tabular}
\end{table}

Additionally, Tomas Chapman Fromm was allocated as Business Analyst, and Ben        % TODO: Add some more waffle here
Lewis as team leaders for website development and back-end engineering
respectively.

\vspace{-4mm}\subsection{Scheduling}\vspace{-2mm}
In-person group meetings Monday and Friday, as well as additional meetings or
collaborative work time to help meet deadlines. Monday meetings to see progress
on weekend tasks and set tasks for the week, Friday meetings to see progress on
week and set tasks for the weekend.

\vspace{-4mm}\subsection{References}\vspace{-2mm}                                   % TODO: Add some more waffle here
References

\end{document}
